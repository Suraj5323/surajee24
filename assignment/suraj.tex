%\iffalse
\let\negmedspace\undefined
\let\negthickspace\undefined
\documentclass[journal,12pt,twocolumn]{IEEEtran}
\usepackage{cite}
\usepackage{amsmath,amssymb,amsfonts,amsthm}
\usepackage{algorithmic}
\usepackage{graphicx}
\usepackage{textcomp}
\usepackage{xcolor}
\usepackage{txfonts}
\usepackage{listings}
\usepackage{enumitem}
\usepackage{mathtools}
\usepackage{gensymb}
\usepackage{comment}
\usepackage[breaklinks=true]{hyperref}
\usepackage{tkz-euclide} 
\usepackage{listings}
\usepackage{gvv}  
\usepackage{tikz}
\usepackage{circuitikz} 
\usepackage{caption}

\def\inputGnumericTable{}                                
\usepackage[latin1]{inputenc}                 
\usepackage{color}                            
\usepackage{array}                            
\usepackage{longtable}                        
\usepackage{calc}                            
\usepackage{multirow}                      
\usepackage{hhline}                           
\usepackage{ifthen}                          
\usepackage{lscape}
\usepackage{amsmath}
\newtheorem{theorem}{Theorem}[section]
\newtheorem{problem}{Problem}
\newtheorem{proposition}{Proposition}[section]
\newtheorem{lemma}{Lemma}[section]
\newtheorem{corollary}[theorem]{Corollary}
\newtheorem{example}{Example}[section]
\newtheorem{definition}[problem]{Definition}
\newcommand{\BEQA}{\begin{eqnarray}}
\newcommand{\EEQA}{\end{eqnarray}}
\newcommand{\define}{\stackrel{\triangle}{=}}
\theoremstyle{remark}
\newtheorem{rem}{Remark}


\begin{document}
%

\bibliographystyle{IEEEtran}





%	\logo{


%	}
\title{JEE ADVANCED}
\author{Suraj Kolluru

EE24BTECH110033
}	

%\title{
%	\logo{Matrix Analysis through Octave}{\begin{center}\includegraphics[scale=.24]{tlc}\end{center}}{}{HAMDSP}
%}


% paper title
% can use linebreaks \\ within to get better formatting as desired
%\title{Matrix Analysis through Octave}
%
%
% author names and IEEE memberships
% note positions of commas and nonbreaking spaces ( ~ ) LaTeX will not break
% a structure at a ~ so this keeps an author's name from being broken across
% two lines.
% use \thanks{} to gain access to the first footnote area
% a separate \thanks must be used for each paragraph as LaTeX2e's \thanks
% was not built to handle multiple paragraphs
%

%\author{<-this % stops a space
%\thanks{}}
%}
% note the % following the last \IEEEmembership and also \thanks - 
% these prevent an unwanted space from occurring between the last author name
% and the end of the author line. i.e., if you had this:
% 
% \author{....lastname \thanks{...} \thanks{...} }
%                     ^------------^------------^----Do not want these spaces!
%
% a space would be appended to the last name and could cause every name on that
% line to be shifted left slightly. This is one of those "LaTeX things". For
% instance, "\textbf{A} \textbf{B}" will typeset as "A B" not "AB". To get
% "AB" then you have to do: "\textbf{A}\textbf{B}"
% \thanks is no different in this regard, so shield the last } of each \thanks
% that ends a line with a % and do not let a space in before the next \thanks.
% Spaces after \IEEEmembership other than the last one are OK (and needed) as
% you are supposed to have spaces between the names. For what it is worth,
% this is a minor point as most people would not even notice if the said evil
% space somehow managed to creep in.



% The paper headers
%\markboth{Journal of \LaTeX\ Class Files,~Vol.~6, No.~1, January~2007}%
%{Shell \MakeLowercase{\textit{et al.}}: Bare Demo of IEEEtran.cls for Journals}
% The only time the second header will appear is for the odd numbered pages
% after the title page when using the twoside option.
% 
% *** Note that you probably will NOT want to include the author's ***
% *** name in the headers of peer review papers.                   ***
% You can use \ifCLASSOPTIONpeerreview for conditional compilation here if
% you desire.




% If you want to put a publisher's ID mark on the page you can do it like
% this:
%\IEEEpubid{0000--0000/00\$00.00~\copyright~2007 IEEE}
% Remember, if you use this you must call \IEEEpubidadjcol in the second
% column for its text to clear the IEEEpubid mark.



% make the title area
\maketitle

\newpage

%\tableofcontents

\bigskip

\renewcommand{\thefigure}{\arabic{figure}}
\renewcommand{\thetable}{\arabic{table}}
%\renewcommand{\theequation}{\theenumi}

%\begin{abstract}
%%\boldmath
%In this letter, an algorithm for evaluating the exact analytical bit error rate  (BER)  for the piecewise linear (PL) combiner for  multiple relays is presented. Previous results were available only for upto three relays. The algorithm is unique in the sense that  the actual mathematical expressions, that are prohibitively large, need not be explicitly obtained. The diversity gain due to multiple relays is shown through plots of the analytical BER, well supported by simulations. 
%
%\end{abstract}
% IEEEtran.cls defaults to using nonbold math in the Abstract.
% This preserves the distinction between vectors and scalars. However,
% if the journal you are submitting to favors bold math in the abstract,
% then you can use LaTeX's standard command \boldmath at the very start
% of the abstract to achieve this. Many IEEE journals frown on math
% in the abstract anyway.

% Note that keywords are not normally used for peerreview papers.
%\begin{IEEEkeywords}
%Cooperative diversity, decode and forward, piecewise linear
%\end{IEEEkeywords}



% For peer review papers, you can put extra information on the cover
% page as needed:
% \ifCLASSOPTIONpeerreview
% \begin{center} \bfseries EDICS Category: 3-BBND \end{center}
% \fi
%
% For peerreview papers, this IEEEtran command inserts a page break and
% creates the second title. It will be ignored for other modes.
%\IEEEpeerreviewmaketitl
\section{Subjective Problems}
\begin{enumerate}
\item Let 'd' be the perpendicular distance from the centre of the ellipse $\frac{x^2}{a^2}+\frac{y^2}{b^2}$=1  to the tangent drawn at a point $\Vec{P}$ on the ellipse.If $\Vec{F_1} and  \Vec{F_2}$ are the two $foci$ of the ellipse,then show that $(P{F_1}-P{F_2})^2=4a^2(1-\frac{b^2}{d^2})$.
\hfill(1995- 5 marks)\break
\newline
\item Points $\Vec{A}$,$\Vec{B}$ and $\Vec{C}$ lie on a parabola $y^2=4ax$.The tangents to the parabola at A,B and C taken in pairs , intersect at points $\Vec{P}$,$\Vec{Q}$ and $\Vec{R}$.Determine the ratios of the areas of triangles ABC and PQR. \hfill(1996- 3 marks)\break\newline
\item From a point $\Vec{A}$ common tangents  are drawn to circle $x^2+y^2=\frac{a^2}{2}$ and parabola $y^2=4ax$.Find the area of the quadrilateral formed by the common tangents,the chord of contact of circle and the chord of contact of parabola.
\hfill(1996- 2 marks)\break\newline
\item A tangent to the ellipse $x^2+4y^2=4$ meets the ellipse $x^2+2y^2=6$ at $\Vec{P}$ and $\Vec{Q}$.Prove that the tangents at $\Vec{P}$ and $\Vec{Q}$ of the ellipse $x^2+2y^2=6$ are at right angles.
\hfill(1997- 5 marks)\break\newline
\item The angle between a pair of tangents drawn from a point $\Vec{P}$ to the parabola $y^2=4ax$ is 45\degree .Show that the locus of point $\Vec{P}$ is hyperbola.
\hfill(1998- 8 marks)\break\newline
\item Consider the family of Circles $x^2+y^2=r^2$,$2<r<5$.If in the first quadrant,the common tangent to a circle of this family and the ellipse $4x^2+25y^2=100$ meets the co-ordinate axes at $\Vec{A}$ and $\Vec{B}$,then find the equation of the locus of the mid-point  of AB.
\hfill(1999- 10 marks)\\
\item Find the co-ordinates of all the points $\Vec{P}$ on the ellipse $\frac{x^2}{a^2}+\frac{y^2}{b^2}$=1, for which the area of the triangle $PON$ is maximum ,where $\Vec{O}$ denotes origin  and $\Vec{N}$,the foot of the perpendicular from O to the tangent P.
\hfill(1999- 10 marks)\break\newline
\item  Let ABC be equilateral triangle inscribed in the circle $x^2+y^2=a^2$.Suppose perpendiculars from A,B,C to the major axis of the ellipse $\frac{x^2}{a^2}+\frac{y^2}{b^2}$=1,$(a>b)$ meets the ellipse respectively,at P,Q,R.so that P,Q,R lie on the same side of major axis as A,B,C respectively .Prove that the normals to the ellipse drawn at the points P,Q and R are concurrent.
\hfill(2000- 7 marks)\break\newline
\item Let $C_1$ and $C_2$ be respectively ,the parabolas $x^2=y-1$ and $y^2=x-1$.Let $\Vec{P}$ be any point on $C_1$ and q be any point on $C_2$.Let $P_1$ and $Q_1$ be the reflections of $\Vec{P}$ and $\Vec{Q}$ respectively with respect to the line y=x.Prove that $P_1$ lies on $C_2$,$Q_1$ lies on $C_1$ and PQ $\geq$min({$PP_1,QQ_1$}).Hence or otherwise determine points $P_0$ and $Q_0$ on the parabolas $C_1$ and $C_2$ respectively such that $P_0Q_0$$\leq$PQ for all pairs od points (P,Q) with P on $C_1$ and Q on $C_2$.
\hfill(2000- 10 marks)\break\newline
\item Let $\Vec{P}$ be a point on the ellipse $\frac{x^2}{a^2}+\frac{y^2}{b^2}$=1 ,$0<b<a$.Let the line parallel to y-axis passing through P meet the circle $x^2+y^2=a^2$ at the point $\Vec{Q}$ such that $\Vec{P}$ and $\Vec{Q}$ are on the same side of x-axis.For two positive real numbers r and s,find the locus of the point $\Vec{R}$ on PQ such that PR : RQ=r:s as $\Vec{P}$ varies over the ellipse.
\hfill(2001- 4 marks)\break\newline
\item Prove that,in an ellipse,the perpendicular from a focus upon any tangent and the line joining the centre of the ellipse to the point of contact meet on the  corresponding directrix.
\hfill(2002- 5 marks)\break\newline
\item Normals are drawn from the point $\Vec{P}$ with slopes $m_1,m_2,m_3$ to the parabola $y^2=4x$.If locus of $\Vec{P}$ with $m_1m_2=\alpha$ is a part of parabola itself then find $\alpha$.
\hfill(2003- 4 marks)\break\newline  
\item Tangent is drawn to parabola $y^2-2y-4x+5=0$ at a point P which cuts the directrix at the point $\Vec{Q}$.A point $\Vec{R}$ is such that it divides QP externally in the ratio 1/2:1.Find the locus of point $\Vec{R}$.
\hfill(2004 - 4 marks)\break\newline
\item Tangents are drawn from any point on hyperbola $\frac{x^2}{9}-\frac{y^2}{4}$=1 to the circle $x^2+y^2=9$.Find the locus of mid-point of the chord of contact .
 \hfill(2005 - 4 marks)\\
\item Find the equation of the common tangent in $1^{st}$ quadrant to the circle $x^2+y^2=16$ and the ellipse $\frac{x^2}{25}+\frac{y^2}{4}$=1.Also find the length of the intercept of the tangent between the coordinate axes.
\hfill{(2005 - 4 marks)\break\newline
\end{enumerate}

\end{document}
